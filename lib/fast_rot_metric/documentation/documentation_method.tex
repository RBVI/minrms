%%%%%%%%%%%%%%%%%%%%%%%%%%%%%%%%%%%%%%%%%%%%%%%%%%%%%%%%%%%%%%%%
%%%  This file was provided as documentation to be           %%%
%%%  distributed along with the source code.                 %%%
%%%  This is a LaTeX2e file                                  %%%
%%% (we got it to compile using the 1994/06/01 distribution) %%%
%%%%%%%%%%%%%%%%%%%%%%%%%%%%%%%%%%%%%%%%%%%%%%%%%%%%%%%%%%%%%%%%
%%%  (As of 11/30/1998 this file is identical to the file    %%%
%%%   stored in ~jewett/paper/fast_orient_metric.tex.)       %%%
%%%%%%%%%%%%%%%%%%%%%%%%%%%%%%%%%%%%%%%%%%%%%%%%%%%%%%%%%%%%%%%%

%\documentstyle[twocolumn,10pt]{report}
\documentstyle{article}
\begin{document}

\begin{appendix}{Appendix}

\begin{center}
An $O(1)$ Orientation Comparison Metric:\\
The RMSD Between Two Orientations of a Rigid Body.
\end{center}

     This section describes a method to calculate the sum-squared-distance
between the positions of a rigid set of $N$ points after being
rotated and translated in two different ways
(the rotations and translations are denoted $R$, $\vec{T}$
and $R'$, $\vec{T'}$).

\underline{Definition:} metric, $\Phi$
\begin{equation}
\Phi(R,\vec{T}, R', \vec{T'}) =
\sum_{\alpha=1}^N
|
(R \vec{x_\alpha} + \vec{T}) -
(R' \vec{x_\alpha} + \vec{T'})
|^2 \label{def:phi}
\end{equation}
where $\vec{x_\alpha} = $ the position of the $\alpha$th point
in question.\\

\underline{Method:}\\
This metric can be calculated with 17 floating-point multiplies,
and 15 additions per comparison, independent of the number of points
being rotated.

\underline{Proof:}

Let $\{ \vec{r_\alpha} \}$,
be the original set of points $\{ \vec{x_\alpha} \}$,
only translated and rotated so that
the principle axis
  %%\mbox{(REFERENCE GOES HERE)}
are now aligned with the
$\hat{x}$, $\hat{y}$, and $\hat{z}$ axis,
and the centroid is located at the origin.
Let ${\cal R}$, $\vec{\cal T}$, ${\cal R'}$, $\vec{\cal T'}$,
be the same rotations and translations expressed in this new coordinate system.
It turns out to be much easier to calculate
$\Phi$ using this new data-set, $\{ \vec{r_\alpha} \}$.
Below is a short detour to review principle axis.
It can be skipped without loss of continuity.
  %%%%\mbox{(REFERENCE GOES HERE)}

\begin{center}
Principle-Axis
%$\{ \vec{r_\alpha} \}$ from $\{ \vec{x_\alpha} \}$
%${\cal R}$, $\vec{\cal T}$, ${\cal R'}$, $\vec{\cal T'}$,
%from
%$R$, $\vec{T}$, $R'$, $\vec{T'}$,
\end{center}

  The principle axis of a set of points, $\{ {\bf \hat{p_i}} \}$
%${\bf \hat{p_1}}$, ${\bf \hat{p_2}}$, ${\bf \hat{p_3}}$
are defined to be the eigenvectors of the matrix $\chi$, where:
\begin{equation}
\chi_{i j} = 
%P_{\nu j} = \lambda_i \delta_{i j}   \label{def:P_and_lambda}
\sum_{\alpha=1}^N 
(x_{\alpha i} - x_{cm \ i})(x_{\alpha j} - x_{cm \ j})  \label{def:chi}
%(x_{\alpha \mu} - x_{cm \ \mu})(x_{\alpha \nu} - x_{cm \ \nu})
%P_{\nu j} = \lambda_i \delta_{i j}   \label{def:P_and_lambda}
\end{equation}
where $\vec{x_{cm}}$ is the location of the centriod
\mbox{($ \vec{x_{cm}} \equiv \sum_{\alpha=1}^N \vec{x_\alpha}/N $)}
and $x_{cm \ j}$ is its $j$th component.
%A description of how to calculate
%${\bf \hat{p_1}}$, ${\bf \hat{p_2}}$, and ${\bf \hat{p_3}}$
%from $\{ \vec{x_\alpha} \}$ can be found in
  %%%%\mbox{(REFERENCE GOES HERE)}.

Note $\chi$ is a symmetric matrix, so its eigenvectors
(the principle axis) form an orthogonal basis.
  %%(REFERENCEGOESHERE).
This means it is allways possible to rotate and
translate our initial coordinate data
$\{ \vec{x_\alpha} \}$, to a new orientation
so that the principle-axis are aligned with the
$\hat{x}$, $\hat{y}$, and $\hat{z}$ axis,
and the centroid at the origin.
We rename the rotated, translated data  $\{ \vec{r_\alpha} \}$.

The transformation between these two sets of data is:
\begin{equation}
\vec{x_\alpha} =
P \vec{r_\alpha} + \vec{x_{cm}} \label{def:r}
\end{equation}
where $P$ is a 3x3 pure rotation matrix (determinant 1)
storing the 3 normalized principle-axis as column vectors
$
{\mbox
P \equiv
\left[
{\bf \hat{p_1}}, {\bf \hat{p_2}}, {\bf \hat{p_3}}
\right]
}
$
  We go to all this trouble because
this new set of points
$\{ \vec{r_\alpha} \}$
has the following desireable properties:
(which you can verify by substituting
 Eq.(\ref{def:r}) into Eq.(\ref{def:chi}))
\begin{equation}
\sum_{\alpha=1}^N \vec{r_\alpha}
  = \vec{0} \label{eq:property_centroid_aligned}
\end{equation}
\begin{equation}
\sum_{\alpha=1}^N r_{\alpha i} r_{\alpha j}
  = \lambda_i \delta_{i j} \label{eq:property_principle_axis_aligned}
\end{equation}
where $r_{\alpha j} = $ the $j$th component of $\vec{r_\alpha}$,
where $\delta_{i j}$ is the Kronecker-Delta function,
(\mbox{ $\delta_{i j} \equiv 1$ if $i = j$, and $0$ otherwise})
and $\lambda_1$, $\lambda_2$, and $\lambda_3$
are the eigenvalues corresponding to 
${\bf \hat{p_1}}$, ${\bf \hat{p_2}}$, ${\bf \hat{p_3}}$.
%and are not to be confused with ``moments of innertia''
%used in rigid-body Newtonian physics.

    In order to express $\Phi$,
in terms of $\{ \vec{r_\alpha} \}$, instead of $\{ \vec{x_\alpha} \}$
we need to modify $R$, $\vec{T}$, $R'$, and $\vec{T'}$.
in order to compensate for the rotation and translation we just applied.
Substituting Eq.(\ref{def:r}) into Eq.(\ref{def:phi}) yeilds:
\begin{equation}
\Phi =
%\sum_{\alpha=1}^N
%|
%R {\bf r_\alpha} + {\bf T} -
%R' {\bf r_\alpha} + {\bf T'}
%|^2 = 
%\Phi({\cal R},{\bf \cal T},{\cal R'}, {\bf \cal T'}) =
\sum_{\alpha=1}^N
|
({\cal R} \vec{r_\alpha} + \vec{\cal T}) -
({\cal R'} \vec{r_\alpha} + \vec{\cal T'})
|^2 \label{eq:phi_in_terms_of_r}
\end{equation}
where:
\begin{center}
${\cal R} \equiv R P$\\
${\cal R'} \equiv R' P$\\
${\cal T}  \equiv P \vec{x_{cm}} + \vec{T}$\\
${\cal T'} \equiv P \vec{x_{cm}} + \vec{T'}$
\end{center}

\begin{center}
Calculating $\Phi$
\end{center}
Suppose\\
$\Delta \vec{r_\alpha} =$ the displacement vector of the
$\alpha$th point between the two orientations\\
and $\Delta r_{\alpha \nu}$ is its $\nu$th component.\\
Re-expressing Eq.(\ref{eq:phi_in_terms_of_r}) in terms of components, we get:
\begin{center}
$\Phi=\sum_{\alpha=1}^N \sum_{\nu=1}^3 (\Delta r_{\alpha \nu})^2$
\end{center}
where
\begin{equation}
(\Delta r_{\alpha \nu})^2 = \\
\left(
 \sum_{i=1}^3
    ({\cal R}_{\nu i} r_{\alpha i} + {\cal T}_\nu)
  - ({\cal R}'_{\nu i} r_{\alpha i} + {\cal T}'_\nu)
\right)
\left(
 \sum_{j=1}^3
    ({\cal R}_{\nu j} r_{\alpha j} + {\cal T}_\nu)
  - ({\cal R}'_{\nu j} r_{\alpha j} + {\cal T}'_\nu)
\right)  \label{def:delta_r_squared}
\end{equation}

%Let's assume that the centroid of the
%points $\{ {\bf x_\alpha} \}$ is located at the origin.
%(i.e. $\sum_{\alpha=1}^N {\bf x_\alpha} = {\bf 0}$)
When we expand the square, we can exploint the property that the
centroid is at the origin, Eq.(\ref{eq:property_centroid_aligned}).
This enables us to throw away the terms with odd powers of $r_{\alpha i}$,
(for example: \mbox{${\cal R}_{\nu i} r_{\alpha i} {\cal T}_\nu$})
since they will be annihilated when we eventually sum over $\alpha$.
Making these changes and collecting equivalent terms,
Eq.(\ref{def:delta_r_squared}) reduces to:

\begin{center}
$
(\Delta r_{\alpha \nu})^2 =
  \sum_{i,j=1}^3 (
  {\cal R}_{\nu i} {\cal R}_{\nu j} r_{\alpha i} r_{\alpha j} -
  2 {\cal R}_{\nu i} {\cal R}'_{\nu j} r_{\alpha i} r_{\alpha j} +
  {\cal R}'_{\nu i} {\cal R}'_{\nu j} r_{\alpha i} r_{\alpha j}
  )
  + ({\cal T}_\nu - {\cal T}'_\nu)^2
$
\end{center}
Summing over $\nu$:
\begin{equation}
\sum_{\nu=1}^3 (\Delta r_{\alpha \nu})^2
 = \sum_{i,j=1}^3 (
  \delta_{i j} r_{\alpha i} r_{\alpha j} -
  2 \sum_{\nu=1}^3 {\cal R}_{\nu i} {\cal R}'_{\nu j} r_{\alpha i} r_{\alpha j} +
  \delta_{i j} r_{\alpha i} r_{\alpha j}
  )
  + \sum_{\nu=1}^3 ({\cal T}_\nu - {\cal T}'_\nu)^2   \label{eq:expand_the_square}
\end{equation}
In this last step, we exploited the orthonormality of the
rotation matrices ${\cal R}$ and ${\cal R}'$)

%Let's additionally limit ourselves to considering cases where
%the principle axis of ${\bf x_\alpha}$ lie along the x,y, and z axis.
%(This means, by definition, that there exist three numbers
% $\lambda_i$, with
% $i \in \{1 \ldots 3\}$,
% such that
%$\sum_{\alpha=1}^N x_{\alpha i} x_{\alpha j} = \lambda_i \delta_{i j}$)
%Neither of these assumptions limit the method's generality since,
%because the principle axis of a rigid body are orthogonal, it is
%allways possible to rotate and translate our coordinate data beforehand
%so that these assumptions are true.
%Since $\Phi$ measures
%only {\em relative} differences between the orientations, this will not 
%effect our results. 
%\footnote{ However, care should be taken to insure that the transformations
%$R$, ${\bf T}$, $R'$, and ${\bf T'}$ are expressed in the coordinate
%system centered at the centroid, whose x,y,z axis point in the direction
%of the principle axis.  If this is not true, complensating rotations and
%translations should be applied to $R$, ${\bf T}$, $R'$, and ${\bf T'}$
%beforehand.}

Now, by summing Eq.(\ref{eq:expand_the_square})
over $\alpha$, and exploiting Eq.(\ref{eq:property_principle_axis_aligned}),
we can simplify Eq.(\ref{eq:phi_in_terms_of_r}) to get:

%This assumption also does not limit the method's generality since we can
%allways rotate our initial data so that the principle axis do lie along
%x,y,z axis and then applying a correcting (inverse) rotation to $R$,
%${\bf T}$, $R'$, and ${\bf T'}$.
%(We can do this because principle axis are allways orthogonal.
% We will review the process of finding principle axis later.)

\begin{center}
$\Phi = 
  \sum_{i,j=1}^3 (
  2 \delta_{i j} \lambda_i \delta_{i j} -
  2 \sum_{\nu=1}^3 {\cal R}_{\nu i} {\cal R}'_{\nu j} \lambda_i \delta_{i j}
  )
  + N \sum_{\nu=1}^3 ({\cal T}_\nu - {\cal T}'_\nu)^2
$
\end{center}
\begin{equation}
= 2 \sum_{i=1}^3\lambda_i (1 - \sum_{\nu=1}^3 {\cal R}_{\nu i} {\cal R}'_{\nu i})
   + N \sum_{\nu=1}^3({\cal T}_\nu - {\cal T}'_\nu)^2     \label{eq:simplified_phi}
\end{equation}
${\bf \Box}$

\underline{Property:}

RMSD($\sqrt{\Phi / N}$)
obeys the\\
\mbox{\bf Triangle Inequality}:
%I believe the next equation is false (and hence commented out):
%\begin{equation}
%\Phi(R,{\bf T}, R'',{\bf T''})
%\leq
%\Phi(R,{\bf T},R',{\bf T'}) + \Phi(R',{\bf T'},R'',{\bf T''})
%\end{equation}
\begin{equation}
\sqrt{\Phi(R,{\bf T}, R'',{\bf T''}) / N}
\leq
\sqrt{\Phi(R,{\bf T},R',{\bf T'}) / N} +
\sqrt{\Phi(R',{\bf T'},R'',{\bf T''}) / N}
\end{equation}


   Proof:
Let \\
${\bf X_\alpha} \equiv R {\bf x_\alpha} + {\bf T}$\\
${\bf X'_\alpha} \equiv R' {\bf x_\alpha} + {\bf T'}$\\

Notice that the set of values $\{ X_{\alpha \nu} \}$ 
(ie. the set of all components of ${\bf X_\alpha}$)
itself forms a ``vector'' in a $3 N$ dimensional space, and:\\
\begin{center}
$ \Phi = \sum_{\alpha, \nu} (X_{\alpha \nu} - X'_{\alpha \nu})^2 $
\end{center}
So $\sqrt{\Phi(R,{\bf T}, R',{\bf T'})}$ is simply
%$\Phi(R,{\bf T}, R',{\bf T'})$ are simply
a straight evenly weighted Euclidian-length of the differences between
the primed and unprimed versions of $X_{\alpha \nu}$,
all the components of all the vectors.
% and Euclidian-length-squared
For any finite-dimensional space of real numbers, Euclidian length,
obeys the triangle innequality.
  %%(I will not show this here.  See REFERENCEGOESHERE for more info.).

%The squared-distance between locations of atoms positioned in each
%orientation obeys the triangle inequality, and the sum-squared-distance
%is just the sum of those distances.  So the sum-squared-distance obeys
%the triangle innequality

%This metric has the advantage of being isotropic, has units of
%distance,  weights the translational and rotational contributions equally,
%obeys the triangle innequality, takes into account the shape of the
%molecule being rotated, and is fast to calculate.

\end{appendix}

\end{document}

